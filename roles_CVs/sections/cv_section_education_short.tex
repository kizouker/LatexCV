\section{Education}
\cventry{2025}{Cloud Computing and Azure Course}{ReDI School of Digital Integration}{}{}{
Hands-on, project-based course in Azure, cloud security, and DevOps. Included certification preparation through real-world case projects.
}
\cventry{2024}{Ethical Hacking Course (completed assessment)}{Udacity}{}{}{
Completed final assessment in December 2024. Hands-on training in reconnaissance, OSINT, spoofing, reverse shells, and OWASP Top 10 vulnerabilities. Worked with tools such as Metasploit, Nmap, and Burp Suite in red team scenarios and exploit simulation. Currently deepening skills through active practice on Hack The Box.
}
\cventry{2023 HT-- 2024 VT}{Google CyberSecurity Fundamentals}{Google/Coursera}{ }{}{} 
\cventry{2023 VT -- Ongoing}{ToastMasters}{Lund}{ }{Public Speaking}{} 
\cventry{2022 HT -- Ongoing}{SRE, Fullstack and Front-end}{Udacity@Palo Alto}{ }{Grafana, Terraform, Kubernetes, AWS}{} 
\cventry{2020}{Pedagogical Drama, Story Telling, Improvisational theatre, Groupe Dynamics}{Malmö University}{Malmö}{}{}      
\cventry{2019-2021}{React.JS (ES6, React, Router, Redux, Native) Nanodegree}{Udacity@Palo Alto}{Remote}{}{}  
\cventry{2016}{Programming and Problem-solving in Python. 2.5 ECTS}{Blekinge Institute of Technology}{Ronneby}{}{}  
\cventry{1996 --2003}{MSc in Computer Science and Engineering. 300 ECTS}
{Linköping Institute of Technology}{Linköping}{}
{Computer/IT-Security, Software production, Artificial Intelligence, 
Object-Oriented Programming, Functional Programming, 
Real time programing, Lisp/C, Advanced computer architecture, Hardware construction, Embedded development with micro-controller
Fritid: tränade jujutsu regelbundet under hela universitetstiden och tog svart bälte, 1 dan i Ju-jutsu Kai år 2000;
Kårarbete: Engagerade mig i sektionsarbete år 2001-2002 som studienämndsordförande, för kvalitetssäkring av utbildningen, samarbete med utbildningsledare,
lektorer, studenter, samt andra styrelsemedlemmar. Detta innebar i princip 20h voluntärarbete under detta år.
}  

\section{Master thesis}
\cvitem{Title}{\emph{On the automatic evolution of an OS kernel using temporal Logic and AOP}}
\cvitem{Supervisors}{Julia Lawall phD, Mario Südholt phD, Gilles Muller phD}
\cvitem{Description}{
In IEEE International Conference on Automated Software Engineering, Montréal, Canada, pages 196 --204, 42 citations}
\url{https://www.researchgate.net/publication/220883611\_On_the_automatic_evolution_of_an_OS_kernel_using_temporal_logic_and_AOP}

\newpage

\cventry{1993--1996}{Naturvetenskapligt program}{Rönneskolan, Ängelholm}{}{}{
Tillval: psykologi, filosofi, företagsekonomi, estetisk verksamhet (musik)\\
Teknisk tillval: programmering, elektronik, mikrodatorer, PC-support\\
Kurser: matematik upp till E, kemi A och B, fysik A och B, franska (B-språk)\\
Specialarbete: 3D- och vektorgrafik på Amiga. Programmering i assembler med fokus på rotationsmatriser, linjedragning och grafikalgoritmer.\\
Fritid: tränade jujutsu regelbundet under hela gymnasietiden
}

\\
\cventry{1990--1993}{Grundskola (högstadiet)}{Kungsgårdsskolan, Ängelholm}{}{}{
Slutbetyg: 4,4 – tredje högst i klassen och högst bland killarna.\\
Tillval: Körsång Musikal\\
Läste extra kurs i assemblerprogrammering på fritiden via gymnasiet.\\
Tilldelades stipendium vid skolavslutningen för goda studieresultat och engagemang.
\\
Fritidsaktiviteter: jujutsu (2 gånger/vecka), tennis (2 gånger/vecka), pianospel.\\
}


% \section{Courses}
% \cvitem{}{Scrum Master (CSM), Softhouse Consulting}
% \cvitem{}{Coaching (ICC/ICF/EMCC), SLH Thailand – L-E Uneståhl}
% \cvitem{}{Nonviolent Communication (NVC), 18 dagar utspritt på ett år}
% \cvitem{}{Förhandling, workshop via FEI}
% \cvitem{}{Praktisk projektledning, Wenell Management}s    
% \cvitem{}{Pedagogiskt drama & improvisation, Malmö universitet}
% \cvitem{}{Front-End: React}
% \cvitem{}{DevOps: CA Lisa/Nolio, CA Technologies}
