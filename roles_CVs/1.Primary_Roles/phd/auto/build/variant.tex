 \documentclass[11pt,a4paper]{article}
\usepackage[utf8]{inputenc}
\usepackage[T1]{fontenc}
\usepackage[margin=2.5cm]{geometry}
\usepackage{titlesec}
\usepackage{enumitem}
\usepackage{setspace}
\usepackage{hyperref}

\setstretch{1.1}
\titleformat{\section}{\bfseries\large}{\thesection.}{1em}{}
\titleformat{\subsection}{\bfseries}{\thesubsection}{1em}{}

\title{Forskningsplan\\[4pt]\large Datadriven produktutveckling, DevOps, DataOps och MLOps för säkra inbyggda system}
\author{Rickard Åberg}
\date{\today}

\begin{document}
\maketitle

\section{Bakgrund och motivation}

Jag har en bred bakgrund inom mjukvaruutveckling och systemintegration, där jag tidigt arbetade med \textbf{Continuous Integration (CI)} och \textbf{Continuous Delivery (CD)} inom telekomsektorn. 
Senare, inom bankindustrin, införde jag \textbf{automatiska deploy-lösningar} (t.ex. CA Lisa) och etablerade \textbf{cross-functional DevOps-team} med fokus på kvalitet, effektivitet och samarbete.

Jag har arbetat som \textbf{projektledare, testledare, Scrum Master och utvecklare} och har både teknisk och organisatorisk erfarenhet av att förbättra utvecklings- och driftsprocesser. 
Min tekniska bas omfattar Windows Server, containerteknologier som Docker samt molntjänster i Azure (certifierad SC-900, AZ-900, pågående AZ-500). 
Jag har en \textbf{Master of Science in Computer Science and Engineering} med profilen \emph{Software Engineering}, vilket gav en stark grund i process- och kvalitetsstyrning.

Jag har följt AI-området sedan 1999 och under senare år fördjupat mig i maskininlärning (Udacity, Kaggle) och egna Python-projekt. 
Jag har experimenterat med \textbf{LLM-baserade agenter}, \textbf{Retrieval-Augmented Generation (RAG)} och \textbf{vektordatabaser}. 
Mitt mål är nu att undersöka hur AI, data och moderna utvecklingsmetoder kan kombineras för att skapa \textbf{robusta, säkra och användarcentrerade produkter} inom fordon- och flygindustrin.

\section{Syfte}

Syftet är att undersöka hur \textbf{data och maskininlärning} kan integreras i utvecklingen av mjukvaruintensiva inbyggda system för att skapa produkter som är:
\begin{itemize}[noitemsep]
  \item datadrivna och adaptiva (lär sig av användning och kontext),
  \item resilienta och säkra (uppfyller krav enligt EU:s \emph{Cyber Resilience Act} och \emph{Data Act}),
  \item effektiva och hållbara,
  \item användarcentrerade (förbättrad infotainment, säkerhet och funktionalitet).
\end{itemize}

Projektet ska bidra med en \textbf{holistisk modell} som förenar \textbf{DevOps, DataOps och MLOps} och inför \textbf{säkerhet som en fjärde aspekt} i hela livscykeln.

\section{Forskningsfrågor}

\begin{enumerate}[noitemsep]
\item Vilken är rollen och användningen av data i utvecklingen av mjukvaruintensiva inbyggda system?
\item Vilka arkitekturer och processer krävs för att förena DevOps, DataOps och MLOps?
\item Hur förändras roller och arbetssätt när AI-baserade komponenter blir en del av produktlivscykeln?
\item Hur kan säkerhets- och integritetsaspekter (DevSecOps, GDPR, CRA) integreras utan att hämma innovationstakten?
\item Hur kan dessa principer omsättas i praktiska pipeline-lösningar för fordon, flyg och IoT?
\end{enumerate}

\section{Metod och genomförande}

\subsection{Empirisk del}
\begin{itemize}[noitemsep]
  \item Intervjuer och workshops med industripartners inom fordon, flyg och IoT.
  \item Fallstudier av miljöer som kombinerar DevOps och AI-komponenter.
  \item Analys av hur data samlas in, taggas och används i pipeline-flöden.
\end{itemize}

\subsection{Teknisk/teoretisk del}
Tre huvudsakliga pipelines:
\begin{enumerate}[noitemsep]
  \item \textbf{DevOps-pipeline} – CI/CD, test och deployment av kod.
  \item \textbf{DataOps-pipeline} – insamling, tvättning, annotering (taggning) och lagring av data, inklusive vektordatabaser.
  \item \textbf{MLOps-pipeline} – träning, validering och distribution av modeller samt återkoppling till utveckling och produktledning.
\end{enumerate}

Dessa kompletteras av ett \textbf{tvärgående säkerhetslager (DevSecOps)} som omfattar:
automatiserad sårbarhetsskanning och penetrationstestning (CVE-baserade verktyg),
dataskydd (GDPR, EU Data Act)
och livscykelspårbarhet samt \emph{security-by-design} enligt \emph{Cyber Resilience Act}.

\subsection{Fallstudie / Proof of Concept}
Exempel: ett videobaserat hastighetsigenkänningssystem för fordon.  
DataOps = insamling och annotering av videodata;  
MLOps = träning av datorseende-modeller;  
DevOps = kontinuerlig integration i inbyggd miljö;  
DevSecOps = automatiska penetrationstester och logg-/integritetsskydd.

\subsection{Deltagande metodik – rollspel och processimulering}

För att empiriskt undersöka samspelet mellan roller, dataflöden och artefakter används en \textbf{deltagande, spelbaserad metodik} inspirerad av dramatpedagogik och teamutveckling.  
Ett \textbf{DevOps-labspel} låter deltagare från företag simulera hur artefakter, data och beslut rör sig genom DevOps-, DataOps- och MLOps-pipelines.  
Syftet är att:
\begin{itemize}[noitemsep]
  \item synliggöra beroenden och flaskhalsar,
  \item experimentera med nya roller och ansvar,
  \item stimulera kreativitet och organisatoriskt lärande.
\end{itemize}
Observationer från simuleringarna används för att validera och förbättra den föreslagna pipeline-modellen.

\section{Förväntade resultat}

\begin{itemize}[noitemsep]
  \item En referensmodell som integrerar DevOps, DataOps, MLOps och DevSecOps.
  \item Rekommendationer för nya roller och arbetssätt.
  \item Arkitekturexempel och visualisering av pipeline-flöden.
  \item En spel- eller simuleringsprototyp som stödjer lärande och analys.
\end{itemize}

\section{Relevans och påverkan}

Projektet möter industrins behov av att kombinera snabb leverans med hög säkerhet.  
Det bidrar till att uppfylla krav i EU:s \emph{Cyber Resilience Act}, \emph{Data Act} och \emph{NIS2}, samt stärker svensk industris förmåga att utveckla säkra, skalbara och hållbara system.

\section{Avgränsningar och vidare perspektiv}

Huvudfokus ligger på process- och arkitekturfrågor kring DevOps, DataOps och MLOps.  
En vidare aspekt är att inkludera de nya EU-reglerna mer systematiskt som en genomgående dimension – enligt principen för \emph{cross-cutting concerns}, där säkerhet och data ses som tvärgående egenskaper som genomsyrar hela utvecklingskedjan.  

En annan tvärgående egenskap i datadriven produktutveckling är \textbf{data-centric logging}.
Traditionell loggning har länge betraktats som en systemövergripande mekanism, men i datadrivna produkter får loggningen
en ny roll: att samla användarhändelser och kontextdata som underlag för analys, modellträning och kontinuerlig förbättring.
Som Bosch (2016) beskriver har utvecklingen gått från funktionscentrerade till datacentrerade processer, där insamling och användning av data
måste integreras i hela mjukvarulivscykeln.
Data-centric logging blir därmed en ny form av cross-cutting concern — en aspekt som förenar DevOps-, DataOps- och MLOps-pipelines.

Denna fjärde pipeline kan ses som parentes, men är avgörande för att bygga säkra, resilienta och framtidssäkra system.

\section{Referenser (preliminära)}

\begin{itemize}[noitemsep]
  \item Bosch, J. \& Holmström Olsson, H. (2019–2024). Artiklar om datadriven produktutveckling och AI-integration.
  \item Bosch, J. (2016). \emph{Speed, Data, and Ecosystems: Excelling in a Software-Driven World}. IEEE Software, 33(1), 82–88.
  \item Bishop, C. (2006). \emph{Pattern Recognition and Machine Learning}. Springer.
  \item Goodfellow, I., Bengio, Y., Courville, A. (2016). \emph{Deep Learning}. MIT Press.
  \item Ng, A. (2018). \emph{Machine Learning Yearning}.
  \item EU Cyber Resilience Act (2024); EU Data Act (2024); NIS2 Directive (2023).
\end{itemize}

\end{document}
