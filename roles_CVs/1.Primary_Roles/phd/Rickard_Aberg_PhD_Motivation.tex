\documentclass[11pt,a4paper]{article}
\usepackage[utf8]{inputenc}
\usepackage[T1]{fontenc}
\usepackage[margin=2.5cm]{geometry}
\usepackage{setspace}
\setstretch{1.2}

\begin{document}

\begin{center}
\Large{\textbf{Letter of Motivation – PhD Position in Computer Science}}\\[4pt]
\large{Rickard Åberg, Malmö, Sweden}
\end{center}

\vspace{0.8cm}

\noindent
Dear Members of the Selection Committee,

\vspace{0.3cm}

I am writing to express my strong interest in the PhD position in \textit{Data-Driven Product Development, Product Improvement and Product Management} at Malmö University. After more than fifteen years in the software industry, I now wish to return to academia to deepen my understanding of how data, AI and automation can transform both product development and organizational processes.

I hold a Master of Science in Computer Science and Engineering from Linköping University, where I conducted my thesis at \'{E}cole des Mines de Nantes. The work resulted in two research papers in aspect-oriented programming and temporal logic, presented at international conferences. This experience provided me with a solid foundation in formal reasoning, architecture and research methodology — perspectives that have guided me throughout my professional career.

In industry, I have worked as developer, test manager, project leader and DevOps consultant for organizations such as Teleca, Cybercom and Bouvet. Around 2012–2013, when I introduced CA Lisa solutions for automated testing at a large bank, I realized that I was working with what would soon be known as DevOps. At Bouvet, I later led a consulting initiative focusing on DevOps and process improvement. These experiences gave me first-hand insight into how process architecture, automation and data flow impact innovation and learning across teams.

My academic curiosity has remained constant. My first course in Artificial Intelligence, already in 1999, was based on Russell and Norvig’s \textit{Artificial Intelligence: A Modern Approach}, which introduced an agent-based view of rational behavior. Those ideas — combining symbolic reasoning with probabilistic learning — continue to fascinate me, especially in the light of today’s LLM-based agents and data-driven decision systems. Since 2018, I have revisited AI through modern courses and personal experiments, from Andrew Ng’s foundational materials to hands-on prototyping using large language models and vector databases.

I enjoy thinking in abstract models and architectures, but also in processes. Concepts like DevOps, DataOps and MLOps intrigue me because they connect the human and technical layers of innovation — how feedback, automation and collaboration create adaptive systems. I also enjoy communicating ideas and connecting people: through Toastmasters, presentations and mentoring, I have developed a passion for translating complex technical concepts into clear and engaging insights.

I see this PhD position as an opportunity to combine my industrial experience with academic rigor. My goal is to contribute to research on how data-driven development and AI can reshape industrial practices — not only by improving algorithms or architectures, but by aligning them with human understanding, organizational structures and long-term sustainability. Malmö University’s strong collaboration with Software Center and its applied research orientation make it an ideal environment for this work.

Thank you for considering my application. I look forward to the opportunity to contribute to your research community.

\vspace{0.6cm}
\noindent
Sincerely,\\[4pt]
\textbf{Rickard Åberg}

\end{document}
