\section*{Bakgrund och tidig forsknings­erfarenhet}
Jag har en bred bakgrund inom mjukvaruutveckling och systemintegration, där jag tidigt arbetade med Continuous Integration inom telekomsektorn. 
När jag fick möjligheten att vara ansvarig för Deploymentprocessen på en bank och införa CA Lisa-lösningar fick jag upp ögonen för DevOps redan 2012 till 2013. På Bouvet drev jag därefter en satsning och konsultgrupp inom DevOps och processutveckling.

Jag genomförde mitt examensarbete vid \'{E}cole des Mines de Nantes och Linköpings universitet, vilket resulterade i forskningsartiklar~\cite{aberg2003aosd,aberg2003ase}. Erfarenheten gav mig en solid grund inom formella metoder och programvaruarkitektur – teman med tydlig koppling till dagens utmaningar inom datadriven mjukvaruutveckling och AI-baserad produktutveckling.

Jag har en Master of Science in Computer Science and Engineering (Software Engineering) samt erfarenhet av DevOps, testledning och systemarkitektur. Sedan 2018 har jag åter fördjupat mig i AI och datadrivna tekniker~\cite{bishop2006pattern,goodfellow2016deep,ng2018yearning,russell1995ai}, särskilt kopplingen mellan LLM-baserade agenter, RAG och vektordatabaser.



\section{Security and Privacy Considerations}
\begin{enumerate}[noitemsep]
\item How can security and privacy aspects (DevSecOps, GDPR, CRA) be integrated without hampering the pace of innovation?
\end{enumerate}
