\documentclass[11pt,a4paper]{article}
\usepackage[utf8]{inputenc}
\usepackage[T1]{fontenc}
\usepackage[swedish]{babel}
\usepackage[margin=2.5cm]{geometry}
\usepackage{setspace}
\setstretch{1.2}

\begin{document}

\begin{center}
\Large{\textbf{Motivationsbrev – Doktorand i Datavetenskap}}\\[4pt]
\large{Rickard Åberg, Malmö, Sverige}
\end{center}

\vspace{0.8cm}

\noindent
\textbf{Till urvalskommittén,}

\vspace{0.4cm}

Jag söker doktorandtjänsten i \textit{Datavetenskap – datadriven produktutveckling, produktförbättring och produktledning} vid Malmö universitet eftersom jag vill bidra till nästa generation av intelligenta och adaptiva mjukvaruintensiva system. Efter mer än femton år i IT-branschen vill jag nu återvända till den akademiska världen för att fördjupa mig i hur data, AI och automatisering kan användas för att skapa mer hållbara och säkra produkter.

Jag är civilingenjör i datateknik från Linköpings universitet och genomförde mitt examensarbete vid École des Mines de Nantes. Arbetet ledde till två publicerade forskningsartiklar om aspektorienterad programmering och temporallogik. Det gav mig en tidig inblick i forskningsmiljöer och väckte en nyfikenhet som nu har vuxit till en tydlig drivkraft att forska vidare.

I industrin har jag haft roller som utvecklare, testledare, projektledare och DevOps-konsult, bland annat på Teleca, Cybercom och Bouvet. Jag införde DevOps-metodik redan innan begreppet blivit etablerat i Sverige, och har i praktiken arbetat med många av de frågor som nu står i centrum för forskningen kring datadriven produktutveckling – till exempel kontinuerlig integration, automatiserade tester, datahantering och lärande system. Jag har även följt utvecklingen inom AI och cybersäkerhet, inklusive hur regelverk som GDPR och EU:s Cyber Resilience Act påverkar utvecklingsprocesser.

Som person är jag analytisk, nyfiken och reflekterande. Jag trivs i miljöer där teknik möter människa, och har ett stort intresse för kommunikation och kunskapsdelning – något jag bland annat utvecklat genom Toastmasters och handledning av kollegor.

Min långsiktiga ambition är att bidra till forskning och undervisning som förenar teori och praktik. Jag vill utveckla metoder och ramverk som gör datadriven och AI-stödd utveckling mer tillförlitlig, transparent och hållbar. Jag tror att Malmö universitet, med sin profil inom tillämpad forskning och sina starka industrisamarbeten, är en utmärkt miljö för detta.

\vspace{0.6cm}
\noindent
Med vänliga hälsningar,\\[4pt]
\textbf{Rickard Åberg}

\end{document}
