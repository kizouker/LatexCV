\documentclass[a4paper,11pt]{letter}
\usepackage[utf8]{inputenc}
\usepackage[T1]{fontenc}
\usepackage[swedish]{babel}
\usepackage{geometry}
\geometry{left=2.5cm,right=2.5cm,top=2.5cm,bottom=2.5cm}
\usepackage{parskip}
\usepackage{lmodern}

% \signature{Rickard Åberg}
\address{Bodekullsgatan 34b\\214 40 Malmö\\rickard.aaberg@icloud.com\\+46 709 43 14 01}
\date{\today}

\begin{document}

\begin{letter}{}
\opening{Hej,}



Jag söker till Kustbevakningens aspirantutbildning eftersom jag vill arbeta i en samhällsviktig verksamhet där säkerhet, ansvar och operativ förmåga står i centrum. Kustbevakningens uppdrag inom sjö- och miljöräddning, tillsyn, brottsbekämpning samt beredskap som en del av totalförsvaret tilltalar mig starkt. Jag motiveras av att arbeta i en miljö där professionalism, samarbete och gott omdöme är avgörande.

Jag är utbildad civilingenjör i datateknik och har lång erfarenhet av strukturerat arbete i tekniska och digitala miljöer. Jag är van vid att arbeta metodiskt med felsökning, avvikelsehantering och dokumentation, samt att följa givna rutiner och instruktioner. I tidigare roller har jag haft ansvar för att analysera komplexa situationer, prioritera åtgärder och säkerställa kvalitet och spårbarhet, även under tidspress.

Jag har lång sjövana från tidig ålder genom segling, segelskola och regelbunden segling med familj. Jag har genomgått kurs för Förarintyg och har därigenom god kunskap inom navigation, sjöregler och sjösäkerhet. Jag är även certifierad PADI Open Water Diver, vilket har gett mig praktisk erfarenhet av säkerhetsrutiner, riskbedömning och att följa tydliga procedurer i miljöer där marginalerna är små.

Under cirka fem säsonger har jag dessutom regelbundet kallbadat utan bastu. Genom detta har jag utvecklat god förståelse för kroppens reaktioner på kyla, inklusive risker som hypotermi och afterdrop. Erfarenheten har lärt mig vikten av kontrollerad exponering, egen observation och återhämtning, samt att alltid respektera fysiologiska gränser. Detta har stärkt mitt säkerhetsmedvetande och mitt lugna, metodiska förhållningssätt i krävande miljöer.

Jag trivs i sammanhang där arbetet präglas av beredskap, tempoväxling och nära samarbete med kollegor. Jag är lugn och stabil i pressade situationer, har god anpassningsförmåga och värdesätter tydlig kommunikation. Jag har ett starkt rättspatos, gott omdöme och hög personlig mognad, och jag är van vid att uttrycka mig väl i tal och skrift på både svenska och engelska.

Jag ser Kustbevakningen som en långsiktig arbetsgivare där min tekniska systemförståelse, mitt säkerhetsmedvetande och min sjövana kan bidra till verksamhetens operativa förmåga. Jag är motiverad att genomgå den utbildning som krävs och att över tid ta ansvar inom myndighetens uppdrag.

Jag ser fram emot möjligheten att presentera mig vidare vid en intervju.


\vspace{1.5em}
\noindent
Med vänliga hälsningar,\\[2em]
Rickard Åberg


\end{letter}

\end{document}
