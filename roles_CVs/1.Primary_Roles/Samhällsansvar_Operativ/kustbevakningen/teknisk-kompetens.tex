\section{Teknisk bakgrund och systemförståelse}

\cvitem{}{
\begin{itemize}
  \item \textbf{Teknisk systemförståelse} – Civilingenjör i datateknik med bred förståelse för hur tekniska system samverkar, inklusive mjukvara, hårdvara, sensorer och styrsystem i operativa miljöer.

  \item \textbf{System- och driftorienterat arbetssätt} – Van vid att arbeta nära tekniska system i drift, med fokus på stabilitet, tillgänglighet och korrekt funktion.

  \item \textbf{Logiskt och strukturerat tänkande} – Stark förmåga att analysera systembeteenden, följa flöden och förstå sekvenser, beroenden och styrlogik.

  \item \textbf{Teknisk felsökning och avvikelsehantering} – Metodisk i identifiering av felorsaker, bedömning av risker och genomförande av åtgärder vid tekniska avvikelser eller driftstörningar.

  \item \textbf{Tekniknära arbete} – Erfarenhet av arbete nära operativsystem, kommunikation mellan system, hårdvara och tekniska gränssnitt.

  \item \textbf{Mätning, noggrannhet och kontroll} – Förstår vikten av precision, verifiering och uppföljning i tekniska och säkerhetskritiska sammanhang.

  \item \textbf{Rutiner och säkerhet} – Van vid att arbeta enligt givna rutiner, checklistor och instruktioner, med högt säkerhetsmedvetande.

  \item \textbf{Dokumentation och rapportering} – Erfaren av teknisk dokumentation, tydlig rapportering och spårbarhet av åtgärder och beslut.

  \item \textbf{Snabb inlärning i nya system} – Snabblärd i nya tekniska system och arbetsmetoder, med fokus på korrekt hantering, kvalitet och säker drift.
\end{itemize}
}
