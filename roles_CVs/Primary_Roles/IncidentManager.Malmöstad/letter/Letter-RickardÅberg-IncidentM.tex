\documentclass[11pt,a4paper]{article}

\usepackage[swedish]{babel}
\usepackage[T1]{fontenc}
\usepackage[utf8]{inputenc}
\usepackage{lmodern}
\usepackage[margin=2.5cm]{geometry}

\begin{document}

\begin{flushright}
Rickard Åberg\\
[Malmö]\\
[rickard.mail@exempel.se] \quad [070-xxx xx xx]\\
\end{flushright}

\vspace{1em}

\noindent \textbf{Till: [Malmö stad / Organisationens namn]}\\
\textbf{Angående: Incident Manager / Incidentkoordinator, ref [ref.nr]}\\
\textbf{Datum: 11 september 2025}

\vspace{1em}

Hej,

Jag söker tjänsten som \textbf{Incident Manager/Incidentkoordinator} hos [Malmö stad/er organisation]. Jag har verkat i roller nära drift och verksamhet i flera år: som \textit{Problem Manager på Länsstyrelsens IT}, \textit{Deployment/Release Manager} på IKANO, i \textit{Application Management-serviceteam} på Telenor, samt tidigare som utvecklare. Jag är civilingenjör i datateknik och har god teknisk förståelse, särskilt inom \textbf{Microsoft Azure} och \textbf{IT-säkerhet}. Kombinationen av processdisciplin (ITIL), teknisk bredd och lugn kommunikation i pressade lägen gör att jag kan leda Major Incidents från första minut till återställning, och därefter säkra lärande och förbättring.

Jag har under flera uppdrag, bland annat på IKANO och Länsstyrelsens IT, hanterat många Major Incidents och arbetat som Problem Manager med rotorsaksanalyser – något jag har stort intresse för och drivs av. Jag har bidragit till att identifiera grundorsaker till återkommande driftstörningar, exempelvis en felaktig DNS-konfiguration som endast fungerade sporadiskt. Min erfarenhet omfattar många år inom ITIL-organisationer och jag har även kombinerat detta med DevOps-tänk, bland annat som konsult där jag ledde satsningar på att införa DevOps-processer.

Jag har främst arbetat som \textbf{Problem Manager}, där jag ansvarat för rotorsaksanalyser och drivit förbättringsarbete efter incidenter. I flera uppdrag, bland annat på IKANO och Länsstyrelsens IT, har jag varit aktiv deltagare vid hantering av Major Incidents och samarbetat nära Incident Managers, Application Management, Operations och Infrastrucdture. Genom detta har jag fått god erfarenhet av att arbeta strukturerat med incidentprocesser, kommunikation och lärande efter allvarliga händelser, t.e.x genom att bjuda in till Retrospectives och fånga upp kunskap och lärande från alla inblandade parter.

t och reducerade antalet involverade resurser från 15 till 2.
  
\paragraph{Varför jag matchar annonsen}
\begin{itemize}\setlength\itemsep{0.3em}
  \item \textbf{Ledning av Major Incidents:} Van att initiera och leda war rooms, fördela åtgärder mellan infrastruktur, applikation och leverantörer, samt hålla tydlig kommunikation till driftledning och verksamhet.
  \item \textbf{ITIL \& förbättring:} Arbetar strukturerat med Incident, Problem och Change. Driver Post-Incident Reviews (PIR) och rotorsaksanalyser (RCA) som omsätts till konkreta åtgärder i backlogg. Följer upp MTTA/MTTR och trender.
  \item \textbf{Tvärfunktionell samverkan \& SIAM:} Vana att navigera mellan interna team och externa parter (\textit{SLA/OLA}) för snabb återställning och tydligt ansvarstagande.
  \item \textbf{Teknisk förståelse:} Azure-miljöer (Compute/Networking/Storage), Entra ID, Defender och Sentinel (SIEM), samt grundläggande SOAR/XDR-flöden. Tidigare utvecklarbakgrund underlättar dialogen med teknikteam.
  \item \textbf{Kommunikation:} Trygg och lugn i kris. Anpassar budskapet till mottagare – från tekniker till verksamhet och ledning – och säkerställer spårbar information i t.ex. ServiceNow/Jira/Confluence.
\end{itemize}

\paragraph{Exempel på resultat}

  \item Som Deployment Manager på IKANO införde jag en \textit{release readiness}-checklista som både minskade tiden för produktionssättning flerfaldig


\paragraph{Vad jag tar med till er}
Jag tar ansvar för hela kedjan: från första larm och prioritering, via koordinerat åtgärdsarbete och transparent kommunikation, till lärande uppföljning och varaktiga förbättringar. Min bakgrund i offentlig sektor ger förståelse för tillgänglighet, informationssäkerhet och spårbarhet, samtidigt som min tekniska bredd hjälper mig att snabbt komma till kärnan vid komplexa incidenter.

Jag ser fram emot att bidra till er stabila och förutsägbara tjänsteleverans, och gärna berätta mer i en intervju. Tack för att ni överväger min ansökan.

\vspace{1em}
Med vänlig hälsning,\\[0.5em]
\textbf{Rickard Åberg}

\end{document}
