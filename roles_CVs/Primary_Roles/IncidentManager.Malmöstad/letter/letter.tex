\documentclass[11pt,a4paper]{article}

\usepackage[swedish]{babel}
\usepackage[T1]{fontenc}
\usepackage[utf8]{inputenc}
\usepackage{lmodern}
\usepackage[margin=2.5cm]{geometry}

\begin{document}

\begin{flushright}
Rickard Åberg \\
Malmö \\
raberg@duck.com \\
070-943\,14\,01 \\
\end{flushright}

\vspace{1em}

\noindent Malmö, 11 september 2025

\vspace{1em}

\noindent \textbf{Till rekryteringsteamet, IT- och digitaliseringsavdelningen, Malmö stad} \\
\textbf{Angående: Tjänsten som Incident Manager (ref. 20251970)}

\vspace{1em}

Hej,

Jag söker tjänsten som \textbf{Incident Manager} på Malmö stads IT- och digitaliseringsavdelning. Med en bakgrund som civilingenjör i datateknik och mångårig erfarenhet från roller inom IT-drift, service management och utveckling, är jag övertygad om att jag kan bidra till ert arbete med att säkerställa stabila och trygga digitala tjänster för staden.

I rollen som \textit{Problem Manager på Länsstyrelsernas IT} ansvarade jag för tvärfunktionella möten kring incident- och problemhantering. Jag samarbetade dagligen med incident managers, service desk, service managers och tekniska specialister inom bland annat nätverk, integration och Windows, samt eskalerade flaskhalsar till processägare och ledning. Genom en lyckad förändring av arkitekturen kunde vi minska antalet månatliga felkoder från 1400 till 50.

Som \textit{Deployment Manager på Ikano Bank} samordnade jag deployment till produktionsmiljöer i komplexa miljöer med BizTalk, SQL Server, SharePoint och CA Lisa/Nolio. Jag ansvarade för att koordinera 15–20 specialister och förbättrade processer som halverade resursbehovet i Operations och höjde kvaliteten markant. På \textit{Telenor} arbetade jag med service management, processdesign och ITIL-processer samt felsökning i högvolymmiljöer. Jag har även erfarenhet som projektkoordinator och kravansvarig på \textit{IKEA IT}, där jag drev internationella projekt i samverkan med både verksamhet och utveckling.

Min erfarenhet gör att jag trivs i pressade situationer där struktur, kommunikation och ledarskap är avgörande. Jag är van vid att koordinera både interna team och externa leverantörer, samt att arbeta med förbättringsarbete enligt ITIL, agila arbetssätt och DevOps-principer. Med min tekniska bredd – från systemintegration och utveckling till molnplattformar och IT-säkerhet – kan jag snabbt förstå tekniska samband och skapa förtroende i dialogen med specialister.

Jag motiveras av att bidra till stabilitet och service i samhällsviktiga system, och ser Malmö stads IT- och digitaliseringsavdelning som en plats där jag kan använda både min tekniska bakgrund och min ledarskapsförmåga för att göra verklig skillnad.

Jag ser fram emot att få berätta mer om hur jag kan bidra till ert team och till Malmö stads digitala infrastruktur.

\vspace{2em}
Med vänlig hälsning, \\[1em]
\textbf{Rickard Åberg}

\end{document}
