\documentclass[11pt,a4paper]{article}

\usepackage[swedish]{babel}
\usepackage[T1]{fontenc}
\usepackage[utf8]{inputenc}
\usepackage{lmodern}
\usepackage[margin=2.2cm]{geometry}
\usepackage{enumitem}
\setlist[itemize]{topsep=3pt,itemsep=3pt,parsep=0pt,left=0pt}

\begin{document}

{\LARGE \textbf{Rickard Åberg} \normalsize \quad Malmö \quad rickard.mail@exempel.se \quad 070-xxx\,xx\,xx}\\[2pt]
\textit{Civilingenjör i datateknik • Incident/Problem Management • DevOps/Deployment • Azure \& IT-säkerhet}

\hrule
\vspace{0.8em}

\section*{Professional Summary}
Erfaren \textbf{Incident/Problem Manager} med bakgrund som \textit{Deployment Manager}, \textit{Project/Service Manager} och utvecklare.
Kombinerar \textbf{ledarskap och processtänk (ITIL, Agile/DevOps)} med \textbf{teknisk bredd} inom integration, drift och automatisering
(Bl.a. BizTalk, MS SQL Server, SharePoint, Windows/Linux, CA Lisa/Nolio, Oracle, CI/CD). Har lett komplexa leveranser och
förbättringsinitiativ i stora organisationer (offentlig sektor och enterprise). Tidig \textbf{AI-förståelse} (första kurs 1999) och
\textbf{daglig användning av AI} i arbetet (voice, screenshots, analys) samt aktiv omvärldsbevakning.

\vspace{0.3em}
\section*{Styrkor}
\begin{itemize}
  \item \textbf{Teknisk bredd \& hands-on:} utveckling, integration, drift, automatisering och moln (Azure).
  \item \textbf{Arkitekturförståelse:} analyserar komplexa IT-landskap och omsätter verksamhetskrav till hållbara lösningar.
  \item \textbf{Ledarskap \& koordinering:} trygg i press — leder tvärfunktionella team, leverantörer och verksamhet.
  \item \textbf{Processtänk:} ITIL (incident/problem/change), Agile/Scrum, DevOps; fokus på lärande och varaktiga förbättringar.
\end{itemize}

\vspace{0.3em}
\section*{Mätbara resultat (urval)}
\begin{itemize}
  \item \textbf{Länsstyrelsernas IT – Problem Manager:} minskade månatliga felkoder från \textbf{1400 till 50} genom ny lösning/arkitektur; samordnade incident- och problemarbete mellan flera serviceområden och eskalerade flaskhalsar till ledning.
  \item \textbf{Ikano Bank – Deployment/DevOps:} halverade \textbf{resursbehovet i Operations} vid deployments via \textbf{CA Lisa/Nolio}; andra team kunde frigöras eller gå i standby. \textbf{7–8x högre kvalitet} i UAT/PROD efter processförbättringar. Koordinerade \textbf{15–20 specialister}; byggde upp nytt team (\textbf{5 personer}) vid flytt av Operations (DE \textrightarrow\ DK).
  \item \textbf{DevOps/CI-initiativ:} införde Scrum i tvärfunktionellt team; etablerade CI med automatiska tester \& code coverage — \textbf{sparmarginal: flera dagars testtid per release}. Initierade Puppet-grupp för automation/standardisering.
\end{itemize}

\vspace{0.3em}
\section*{Erfarenhetsöversikt (urval)}
\begin{itemize}
  \item \textbf{Problem Manager — Länsstyrelsernas IT:} ITIL problem/incident, tvärfunktionell samverkan (Webb, Nät, Print, GIS, Windows, Integration), PIR/RCA, rapportering till processägare/ledning.
  \item \textbf{Deployment Manager / Projektledare — Ikano Bank:} CA Lisa/Nolio, BizTalk, MS SQL, SharePoint, Windows/Linux, F5; planering/go-live, riskanalys UAT/PROD, BizTalk360, standups \& Kanban.
  \item \textbf{Process/Service/DevOps — Telenor:} processdesign (ADSL/VDSL), ITIL/SLA servicedesk, felsökning i högvolym, JIRA-plugins, CI (CruiseControl, Emma), integration (J2EE, Tibco, JMS, XSLT).
  \item \textbf{Project \& Product — IKEA IT:} global rollout (30 länder/250 butiker), krav/scope, testledning (8 superusers), DFA/Isell/IHP (NITF, .NET/VB), offshore-koordination, CTE/PPE.
  \item \textbf{QA Manager — Sony Mobile:} kvalitetsdokumentation, testframework, ledde Javautvecklare för testautomation; säkrade leverans med hög kvalitet.
\end{itemize}

\vspace{0.3em}
\section*{Utbildning \& övrigt}
\begin{itemize}
  \item \textbf{Civilingenjör i datateknik} (MSc Computer Science and Engineering).
  \item Första universitetskurs i \textbf{AI} \textbf{1999}; följer och använder AI dagligen (röst, bild/skärm, analys).
  \item Språk: \textbf{svenska} och \textbf{engelska} — mycket god kommunikationsförmåga i tal och skrift.
\end{itemize}

\vspace{0.3em}
\section*{Referens}
\textbf{Jonas Paulsson}, Funktionschef Kompetenscenter, Länsstyrelsernas IT-enhet — 010-224\,45\,89.

\end{document}
