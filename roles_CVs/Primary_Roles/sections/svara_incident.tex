\section*{Motivation och lämplighet för rollen som Incident Manager}

Jag söker rollen som \textbf{Incident Manager} på Malmö stads IT- och digitaliseringsavdelning eftersom jag trivs i pressade situationer, är snabb att agera vid incidenter och vill bidra till stabila digitala tjänster för samhällsviktiga verksamheter.

\textbf{Erfarenhet som matchar annonsen:}
\begin{itemize}
  \item \textbf{ITIL problem} Problem Manager på Länsstyrelsernas IT – samordnade incident- och problemprocesser mellan flera serviceområden, identifierade flaskhalsar och minskade felkoder från \textbf{1400 till 50 per månad} genom ny lösning.
  \item \textbf{Deployment \& drift:} Deployment Manager på Ikano Bank – ansvarade för produktion, koordinerade \textbf{15–20 specialister} vid produktionssättningar och införde CA Lisa/Nolio vilket \textbf{halverade resursbehovet i Operations} och förbättrade kvaliteten i UAT/PROD med \textbf{7–8 gånger färre fel}.
  \item \textbf{DevOps \& helhetsförståelse:} Införde Scrum och CI/CD i tvärfunktionella team; implementerade automatiserade tester och code coverage vilket \textbf{sparade flera dagars testtid per release}. Startade även en Puppet-grupp för automation och kunskapsdelning.
  \item \textbf{Kommunikation och ledarskap:} Ledde kunskapsöverföring när Operations flyttade från Tyskland till Danmark och byggde upp ett nytt team på \textbf{5 personer}. Höll retrospektiv efter varje deployment och drev förbättringar i mål. 
\end{itemize}

\textbf{Styrkor:} Jag ser snabbt helheten, är driven, tar kontakt med andra och tar ansvar för att skapa förbättringar. Med min kombination av teknisk bredd, arkitekturförståelse, processtänk och ledarskap kan jag bidra till att utveckla Malmö stads incidenthantering.
