% Rickard Åberg - Cybersecurity & Ethical Hacking
% LaTeX CV focusing on Security-related roles (urval av erfarenheter)

% This version fixes compilation issues by removing the leftover $1 and adding \end{document}

\documentclass[11pt,a4paper,sans]{moderncv}
\moderncvstyle{banking}
\moderncvcolor{red}

\usepackage[scale=0.75]{geometry}
\usepackage[utf8]{inputenc}
\usepackage[T1]{fontenc}
\usepackage[swedish]{babel}
\usepackage{lmodern}

\name{Rickard}{Åberg}
\title{Cybersecurity & Ethical Hacking}
\address{Bodekullsgatan 34b}{21440 Malmö}{Sverige}
\phone[mobile]{+46~709~43~14~01}
\email{raberg@duck.com}
\social[linkedin]{rickard-åberg-9866891}
\social[github]{kizouker}

\begin{document}
\makecvtitle

\section{Erfarenhet (Urval)}

% ---- 2023 -- Present ----
\cventry{2024-- pågående}{Cybersecurity \& Ethical Hacking}{Självstudier och projekt}{}{}{
\begin{itemize}
\item Praktisk penetrationstestning, nätverksforensik
\item Reverse engineering av firmware, BIOS-säkerhet
\item Etisk hackning av android, iphone och windows
\end{itemize}}

% ---- 2022 ----
\cventry{2024}{Udacity Ethical Hacking (utbildning, del av Accenture)}{Udacity}{Distans}{}{
\begin{itemize}
\item Praktiskt arbete med Metasploit, John the Ripper, Burp Suite och Wireshark
\item Projektuppgifter med sårbarhetsanalys och intrångstest
\end{itemize}}


% ---- 2021--2022 ----
\cventry{2021--2022}{Fullstack Developer}{ADB Safegate}{Malmö}{}{
\begin{itemize}
\item Angular, TypeScript, Azure
\item Arbetade enligt SAFe (Scaled Agile Framework)
\end{itemize}}

% ---- 2012 -- 2019 ----
\cventry{2012--2019}{Konsult / DevSecOps \& Deploy}{Ikano Bank, Nordea, Länsstyrelsen, Verisure}{Malmö med omnejd}{}{
\begin{itemize}
    \item Verisure (2017--2018): Senior Software Engineer med GDPR-ansvar, hantering av persondata och regelefterlevnad
    \item Länsstyrelsen (2015--2016): Problem Manager med ansvar för incidenter och ITIL-processer, minskade felkoder radikalt
    \item Nordea (2014--2015): DevOps, infrastrukturprojekt och IaaS (AWS), initierade puppet-automation
    \item Ikano Bank (2012--2014): Fokus på säkerhet inom CI/CD, DevSecOps och crossfunctional samarbete
\end{itemize}}

% ---- 2004 -- 2012 ----
\cventry{2004--2012}{Systemutvecklare och Integrationsspecialist}{Teleca / Cybercom}{Malmö}{}{
\begin{itemize}
\item Logganalys, systemövervakning, felsökning
\item Utveckling i alla lager: frontend, backend, messaging, databaser
\end{itemize}}

\section{Utbildning}
\cventry{2024}{Bertellmans Scholarship @ Udacity for Ethical Hacking}{Udacity}{Distans}{Praktisk penetrationstestning}{
    \begin{itemize}
        \item Ansökan påbörjad 2023, antagen 2024
        \item Första provet godkänt, men jag repeterar kursmaterialet (del 2 planerad)
    \end{itemize}}
\cventry{2023}{Google Cybersecurity (Distans)}{}{}{}{En översiktskurs i cybersäkerhet för att orientera mig i området. Fokuserar på grundläggande begrepp, hotdetektering och incidenthantering.}


\cventry{2003}{MSc i Datateknik}{Linköpings Universitet}{Linköping}{}{Inriktning mot IT-säkerhet, OS-utveckling och nätverk}
\\\emph{Examensarbete: "Evolving an OS kernel using temporal logic and AOP" (Linux-säkerhet).}}

\section{Certifieringar}
\cvlistdoubleitem{Udacity Ethical Hacking}{Scrum Master}
\cvlistdoubleitem{ISEB / ISTQB (Test)}{React Nanodegree}

\section{Tekniska färdigheter}
\cvlistdoubleitem{Penetrationstestning}{Metasploit, Burp Suite, Wireshark}
\cvlistdoubleitem{Operativsystem}{Linux (Ubuntu), Windows, macOS, Android, iOS}
\cvlistdoubleitem{Säker kodgranskning}{OWASP Top 10, SQL-injection, XSS}
\cvlistdoubleitem{Logganalys/övervakning}{Felsökning, prestandaoptimering, logghantering}
\cvlistdoubleitem{Modern webbutveckling}{React, Angular, TypeScript}

\section{Språk}
\cvitemwithcomment{Svenska}{Modersmål}{}
\cvitemwithcomment{Engelska}{Professionell nivå}{}
\cvitemwithcomment{Franska}{Professionell nivå}{}

\end{document}
