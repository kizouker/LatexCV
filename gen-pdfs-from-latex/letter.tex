\documentclass[a4paper,12pt]{letter}
\usepackage[utf8]{inputenc}
\usepackage{geometry}
\usepackage{graphicx}
\usepackage{amsmath}

% Anpassa sidmarginaler
\geometry{top=30mm, bottom=30mm, left=25mm, right=25mm}

\signature{Rickard} % Din underskrift
\address{Din Adress \\ Din Stad, Postnummer} % Din adress
\date{\today} % Dagens datum

\begin{document}

% Mottagarens uppgifter
\begin{letter}{Arbetsgivarens Namn \\ Företagets Namn \\ Adress \\ Postnummer, Stad}

\opening{Hej [Rekryterarens namn],}

Jag är en civilingenjör inom IT och cybersäkerhet med en stark bakgrund inom systemutveckling, penetrationstester och nätverkssäkerhet. Mitt intresse för cybersäkerhet har funnits sedan tidigt, men har intensifierats de senaste åren i takt med fler attacker och när jag lyssnat på Prime Crime och sett hur nödvändigt området är. Jag har aktivt byggt på min kompetens genom både självstudier och praktiska projekt, inklusive arbete med penetrationstester, forensik och säkerhetsanalys.

Jag har en förmåga att snabbt ta till mig nya områden och applicera min bakgrund inom operativsystem, applikationslager och tekniker jag har använt sedan jag började med datorer. Jag kan visualisera komplexa system och säkerhetsutmaningar för att hitta lösningar som förstärker den övergripande säkerheten.

Under min karriär har jag genomfört säkerhetsgranskningar på flera framstående företag som Nordea och IKEA IT, där jag fokuserade på deras övergång till AWS/cloud och DevSecOps. Jag har också arbetat med säkerhetsanalyser på Ikano, Länsstyrelsen (spam skydd) och Verisure (GDPR). Jag har använt verktyg som Metasploit, Burp Suite och Kali Linux för att identifiera och åtgärda sårbarheter i system och har genomfört avancerade penetrationstester och säkerhetsgranskningar av webbtjänster och nätverk.

Jag trivs i en miljö där jag får använda både mina analytiska färdigheter och kreativitet för att lösa komplexa problem och utveckla innovativa säkerhetslösningar. Jag har även en vana att arbeta metodiskt och noggrant, samtidigt som jag anpassar mig snabbt till nya tekniska utmaningar.

Jag ser fram emot möjligheten att diskutera hur jag kan bidra till ert team och hjälpa er att stärka säkerheten i era system. Jag är tillgänglig för en intervju enligt ert önskemål och kan kontaktas på iMessage (nummer som ovan) eller Signal. Jag delar gärna min Signal QR-kod vid behov.

Min primära e-postadress är raberg@duck.com, som vidarebefordras till min iCloud. Jag är tillgänglig via denna e-post för allmän kommunikation.

Tack för din tid och jag ser fram emot att höra från dig.

\closing{Med vänliga hälsningar,}

\ps{P.S. Jag bifogar även min CV för ytterligare information om min bakgrund och erfarenhet.}

\end{letter}

\end{document}

